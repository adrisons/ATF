Durante la ejecución del programa se pueden dar errores relacionados con los ficheros \texttt{result} y \texttt{checkpoint}.
Por ejemplo, si tratamos con matrices de gran volumen, podría pasar que se guardara el estado en \texttt{checkpoint} pero que un error externo interrumpiese mientras se guarda en el fichero \texttt{result} y tener valores incorrectos.\\
Con el objetivo de reducir la posibilidad de este error es por lo que se guarda cada posición calculada de la matriz resultado en \texttt{result}, en vez de almacenar toda la matriz resultado en cada iteración. De este modo las escrituras son verdaderamente rápidas, ya que sólo almacenan un número en cada iteración.
Para añadir más tolerancia a fallos, se comprueba además la matriz resultado almacenada hasta el momento. Si el nº de filas y de columnas coinciden con los datos del checkpoint, se continúa como antes, si no, se continúa en donde se ha quedado la matriz resultado. 
