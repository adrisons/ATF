\documentclass[a4paper]{article}
\usepackage[T1]{fontenc}
\usepackage[utf8]{inputenc}
\usepackage[spanish]{babel}
\usepackage{mathtools}
\usepackage{amsmath}
\usepackage{graphics}
\usepackage{multicol}
\usepackage{listings}
\usepackage{color, colortbl}
\usepackage{amsfonts}

\newcommand{\HRule}{\rule{\linewidth}{0.5mm}}


\definecolor{mygreen}{rgb}{0,0.6,0}
\definecolor{mygray}{rgb}{0.5,0.5,0.5}
\definecolor{mymauve}{rgb}{0.58,0,0.82}

\definecolor{LightCyan}{rgb}{0.88,1,1}

\lstset{ %
	backgroundcolor=\color{white},   % choose the background color; you must add \usepackage{color} or \usepackage{xcolor}
	basicstyle=\footnotesize,        % the size of the fonts that are used for the code
	breakatwhitespace=false,         % sets if automatic breaks should only happen at whitespace
	breaklines=true,                 % sets automatic line breaking
	captionpos=b,                    % sets the caption-position to bottom
	commentstyle=\color{mygreen},    % comment style
	deletekeywords={...},            % if you want to delete keywords from the given language
	escapeinside={\%*}{*)},          % if you want to add LaTeX within your code
	extendedchars=true,              % lets you use non-ASCII characters; for 8-bits encodings only, does not work with UTF-8
	frame=single,                    % adds a frame around the code
	keywordstyle=\color{blue},       % keyword style
	language=C,                 % the language of the code
	morekeywords={*,...},            % if you want to add more keywords to the set
	numbers=left,                    % where to put the line-numbers; possible values are (none, left, right)
	numbersep=5pt,                   % how far the line-numbers are from the code
	numberstyle=\tiny\color{mygray}, % the style that is used for the line-numbers
	rulecolor=\color{black},         % if not set, the frame-color may be changed on line-breaks within not-black text (e.g. comments (green here))
	showspaces=false,                % show spaces everywhere adding particular underscores; it overrides 'showstringspaces'
	showstringspaces=false,          % underline spaces within strings only
	showtabs=false,                  % show tabs within strings adding particular underscores
	stepnumber=1,                    % the step between two line-numbers. If it's 1, each line will be numbered
	stringstyle=\color{mymauve},     % string literal style
	tabsize=2,                       % sets default tabsize to 2 spaces
	title=\lstname                   % show the filename of files included with \lstinputlisting; also try caption instead of title
}

\begin{document}

% Título
%	\begin{titlepage}
		\begin{center}

			\HRule \\[0.4cm]
			{ \huge \bfseries Codificación y decodificación de códigos usando códigos \texttt{Checksum} y \texttt{Berger}}\\[0.4cm]
			\HRule \\[0cm]

			\vspace{1cm}
			\textsc{\Large Arquitecturas Tolerantes a Fallos}\\[0.5cm]
			\textsc{\Large Curso 2012/2013}\\[0.5cm]

		\end{center}

		\begin{center}
		Pereira Guerra, Adrián \texttt{<adrian.pereira@udc.es>}\\
		https://github.com/adrisons/ATF
		\end{center}
		\vspace{2cm}

%	\end{titlepage}
% Índices

%\tableofcontents
%\vspace{3cm}
%\clearpage


%\section{Introducción}
%	He decidido implementar el checkpoint mediante dos ficheros:
\begin{description}
	\item [\texttt{result.csv}] Que almacena cada posición de la matriz resultado en formato csv
	\item [\texttt{checkPoint.txt}] Que va almacenando la última posición almacenada en \texttt{result.csv}
\end{description}

Por cada posición que se calcula de la matriz resultado, su valor se almacena en el fichero resultado y su posición en el fichero de checkpoint.

El fichero de checkpoint sólo existe mientras el calculo de la matriz resultado no se ha realizado por completo y, además, almacena el path de las dos matrices que se están multiplicando, para controlar que sólo se realice la recuperación si las matrices a multiplicar son las mismas.

Si la ejecución termina inesperadamente, cuando se intenta ejecutar el programa con las mismas matrices, se restaura el estado del sistema utilizando el checkpoint y se continúa ejecutando.

\section{Berger}
	Sabemos que el número de bits que añade Berger es $k = [log_2(I+1)]$ siendo k la longitud de bits añadidos e I la longitud del dato original no codificado. Sin embargo, al decodificar, sabemos que el código codificado tiene longitud $I + k$, es decir, $I + log_2(I+1)$ y, de esta fórmula no se puede despejar I. 
	Por lo tanto, estudio los datos de la Figura \ref{t_berger} para encontrar una relación que me permita conseguir la longitud del dato original a partir del codificado.
	
	\begin{figure}
		\begin{tabular}{| l | c | r |}
				\hline
				Dato original & Añadido & Total \\ \hline
				\rowcolor{LightCyan}
				1 & 1 & 2 \\ \hline
				2 & 1 & 3 \\ \hline
				\rowcolor{LightCyan}
				3 & 2 & 5 \\ \hline
				4 & 2 & 6 \\ \hline
				... & 2 & ... \\ \hline
				\rowcolor{LightCyan}
				7 & 3 & 10 \\ \hline
				8 & 3 & 11 \\ \hline
				9 & 3 & 12 \\ \hline
				... & 3 & ... \\ \hline
				\rowcolor{LightCyan}
				15 & 4 & 19 \\ \hline
				16 & 4 & 20 \\ \hline
				... & 4 &  \\ \hline
				\rowcolor{LightCyan}
				31 & 5 & 36 \\ \hline
				... & 5 & ... \\ \hline
				\rowcolor{LightCyan}
				63 & 6 & 69 \\ \hline
				... & 6 & ... \\ \hline
				\rowcolor{LightCyan}
				127 & 7 & 134 \\ \hline
				... & ... & ... \\ 
				\hline
		\end{tabular}

		\label{t_berger}
		\caption{Tabla de código Berger}
	\end{figure}
	Como se ve en la tabla, cuando el dato original tiene como longitud $2^i - 1$ se añade un nuevo bit a la codificación. Por lo tanto, se cumple que el $ 2^{Anhadido} <= Total < 2^{Anhadido+1}$, y sólo hay que calcular la potencia de dos a la que corresponde el Total para hallar el número de bits del dato original.

%\section{Problemas}
%	Durante la ejecución del programa se pueden dar errores relacionados con los ficheros \texttt{result} y \texttt{checkpoint}.
Por ejemplo, si tratamos con matrices de gran volumen, podría pasar que se guardara el estado en \texttt{checkpoint} pero que un error externo interrumpiese mientras se guarda en el fichero \texttt{result} y tener valores incorrectos.\\
Con el objetivo de reducir la posibilidad de este error es por lo que se guarda cada posición calculada de la matriz resultado en \texttt{result}, en vez de almacenar toda la matriz resultado en cada iteración. De este modo las escrituras son verdaderamente rápidas, ya que sólo almacenan un número en cada iteración.
Para añadir más tolerancia a fallos, se comprueba además la matriz resultado almacenada hasta el momento. Si el nº de filas y de columnas coinciden con los datos del checkpoint, se continúa como antes, si no, se continúa en donde se ha quedado la matriz resultado. 


%\clearpage
%\section{Conclusiones}
%	\input{./conclusiones}

\end{document}
